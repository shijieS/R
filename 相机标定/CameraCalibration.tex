\documentclass[10pt,a4paper]{paper}
\usepackage[utf8]{inputenc}
\usepackage{amsmath}
\usepackage{amsfonts}
\usepackage{amssymb}
\usepackage{makeidx}
\usepackage{graphicx}
\usepackage{ctex}
\usepackage{float}
\usepackage{tikz}
\usepackage{mdframed}
\usepackage{tcolorbox}
\usepackage{listings}
\usepackage{xcolor}
\usepackage{cite}
\usepackage{hyperref}
\author{shijieSun}
\title{三维相机的姿势快速估计方法}

\usepackage{Sweave}
\begin{document}
\Sconcordance{concordance:CameraCalibration.tex:CameraCalibration.Rnw:%
1 19 1 1 0 12 1}

	\maketitle
	\begin{abstract}
		鉴于深度相机(比如:Kinect、Primesense等)被用于各种云台上,被用于车辆的识别、三维建模、行人检测,深度相机需要进行标定,不过,多数情况下,使用传统的标定方法(如点标定、线标定、面标定)对深度相机进行标定,往往忽视了深度相机所提供的信息;深度相机可以提供场景中的点到相机的距离信息,如果有些利用这些信息,将会极大的简化标定过程,使其可以自动化地进行标定;本文提出一种可以用于估计相机外参的深度相机自动标定方法,它可以自动地检测地平面,利用相机的已知内参,计算相机的高度,然后,根据规定的坐标系规则,估计相机的在规定坐标系的外参;如果场景中的地面居于主体地位,本算法可以自动地完成相机标定的流程,如果场景中存在部分地面,需要手动框选地面,实现相机的标定;在标定过程中,不需要输入相机的实际高度或者标定点。
	\end{abstract}
	
	\section{介绍}
	深度相机,即RGB-D相机,是一种可以采集场景中的目标到图像中的信息,目前主流的深度相机包括:Kinect、PrimeSense等;RGB-D相机采集的图片称为深度图,深度图用于表示场景中的目标点到相机的距离的信息;正如其名称,RGB-D相机往往会附带着RGB信息,且在相机出厂之前,已经做了RGB图和深度图之间的对齐调整,且会附带着相机的参数。\\
	
	在进行相机标定的时候,由于相机厂家已经提供了相机的内参数,那么只需要对相机的外参数进行估计,即可完成相机的标定,对于相机的外参数进行估计的过程,我们称之为“相机姿势估计”。需要轨迹的相机参数包括:偏航角($\psi$)、俯仰角($\varphi $)和侧倾角($\theta $)[]\\
	
\end{document}
